\documentclass{../lista}

\begin{document}
	\cabecalhoAlt
		{Simulado | 2º Intensivo para a OBA}
		{\red{Gabarito}}
		{Gabriel Lucena e Iago Mendes}

	\red{Observação: \begin{itemize}
		\item As alternativas das perguntas deste gabarito não estão na mesma ordem do simulado.
	\end{itemize}}

	\begin{secao}{Questões de Astronomia}
	\end{secao}

	\begin{secao}{Questões de Astronáutica}
	\end{secao}

	\begin{secao}{Questões Avançadas}
		\begin{questao}{(1 ponto)}
			Um fenômeno muito conhecido é o da ``laçada de Marte", em que o planeta Marte subitamente muda sua direção de deslocamento no céu, e quando acompanhado por vários dias parece se locomover formando um laço no céu.

			\begin{pergunta}{(1 ponto)}
				Quais planetas, além de Marte, reproduzem o mesmo fenômeno de modo que possamos observá-los em uma noite de céu limpo?

				\red{\begin{itemize}
						\item Todos os planetas reproduzem esse fenômeno. Então, a pegadinha da questão é você marcar somente os planetas que são observáveis à noite, excluindo assim os planetas inferiores (Mercúrio e Vênus), os quais estão sempre próximos ao Sol na Esfera Celeste.
				\end{itemize}}

				\begin{multicols}{3}
					\begin{alternativas}
						\item[$(\quad)$] Mercúrio
						\item[$(\quad)$] Vênus
						\alternativaMarcada Júpiter
						\alternativaMarcada Saturno
						\alternativaMarcada Urano
						\alternativaMarcada Netuno
					\end{alternativas}
				\end{multicols}
			\end{pergunta}
		\end{questao}

		\begin{questao}{(1 ponto) [USAAAO 2021 adaptada]}
			O cometa C/2020 F3 (NEOWISE) atingiu o periélio pela última vez em 3 de julho de 2020. O cometa NEOWISE tem um período orbital de $\approx 4.400$ anos e sua excentricidade é de $0,99921$.

			\begin{pergunta}{(1 ponto)}
				Qual é a distância do periélio do cometa NEOWISE, em $UA$?

				\red{\begin{itemize}
					\item Usando a 3ª Lei de Kepler com unidades do Sistema Solar (anos, $UA$ e massas solares), nós temos:
						\begin{equation}
							\frac{T^2}{a^3} = 1 \quad \therefore \quad a = \sqrt[3]{T^2} = \sqrt[3]{4400^2} \approx 268,5 \; UA
						\end{equation}
					\item Agora, basta calcular a distância do periélio usando a geometria de elipses:
						\begin{equation}
							P = a(1-e) = 268,5 (1-0,99921) \approx 0.212 \; UA
						\end{equation}
				\end{itemize}}

				\begin{alternativas}
					\item $0,0123 \; UA$
					\alternativaMarcada $0,212 \; UA$
					\item $2,69 \; UA$
					\item $26,8 \; UA$
				\end{alternativas}
			\end{pergunta}
		\end{questao}

		\begin{questao}{(1 ponto)}
			Deneb é uma estrela de tipo espectral A2 cuja magnitude aparente na banda V é de 1,25. Certa noite Deneb se divide em 2 novas estrelas com a mesma temperatura da inicial. \\ \\
			Dado: \\
			$\log(2) \approx 0,3$

			\begin{pergunta}{(1 ponto)}
				Qual a nova magnitude aparente na banda V do sistema?

				\red{\begin{itemize}
					\item Como o volume de cada uma das novas estrelas deve ser metade de Deneb, temos a seguinte relação entre os raios das novas estrelas ($R'$) e de Deneb ($R_0$):
						\begin{equation}
									V'=\frac{V_0}{2} \quad \therefore \quad \frac{4\pi R'^3}{3}=\frac{1}{2} \frac{4 \pi R_0^3}{3} \quad \therefore \quad \frac{R'}{R_0}=\sqrt[3]{\frac{1}{2}}
						\end{equation}
					\item Lembrando que $T'=T_0$, podemos usar a equação de Stefan-Boltzmann para encontrar a razão entre as luminosidades:
						\begin{equation}
							\frac{L'}{L_0}=\frac{4 \pi R'^2 \sigma T'^4}{4 \pi R_0^2 \sigma T_0^4}=\left(\frac{R'}{R_0}\right)^2=2^{-\frac{2}{3}}
						\end{equation}
					\item Calculando a razão dos fluxos recebidos, temos:
						\begin{equation}
							\frac{F'}{F_0}=\frac{2L'}{L_0}=2 \cdot 2^{-\frac{2}{3}}=\sqrt[3]{2}
						\end{equation}
					\item Finalmente, usando a relação de Pógson, temos:
						\begin{equation}
							m'-m_0=2,5 \log \left(\frac{F_0}{F'}\right) \quad \therefore \quad m'=2,5 \log \left(\frac{F_0}{F'}\right)+m_0\\
							\therefore \quad m'=2,5 \log \left(2^{-\frac{1}{3}}\right)+1,25=-\frac{2,5 \cdot \log (2)}{3}+1,25 \\
							\therefore \quad m'=-0,25 +1,25 =1,0
						\end{equation}
				\end{itemize}}

				\begin{alternativas}
					\item $1,25$
					\item $2,5$
					\alternativaMarcada $1,0$
					\item $2,0$
				\end{alternativas}
			\end{pergunta}
		\end{questao}

		\begin{questao}{(1 ponto) [USAAAO 2020 adaptada]}
			Em abril de 2020, o \textit{Event Horizon Telescope} divulgou a primeira imagem do buraco negro supermassivo da galáxia \textit{M87}. O buraco negro tem um diâmetro de aproximadamente $270 \; UA$ e está localizado a uma distância de $16,4 \; Mpc$.

			\begin{pergunta}{(1 ponto)}
				No comprimento de onda observado de $1,3 \; mm$, qual é a linha de base mínima aproximada, ou diâmetro efetivo, necessária para a imagem do buraco negro?

				\red{\begin{itemize}
					\item Calculando a resolução angular necessária para observar o buraco negro de \textit{M87}, temos:
						\begin{equation}
							\theta = \frac{270 \; UA}{16,4 \; Mpc} = \frac{270 \; UA}{16,4 \cdot 10^6 \cdot 206.265 \; UA} \approx 7.98 \cdot 10^{-11} \; rad
						\end{equation}
					\item Agora, basta usar a equação de resolução angular de telescópios para encontrar o diâmetro efetivo:
						\begin{equation}
							\theta = 1,22 \frac{\lambda}{D} \quad \therefore \quad D = 1,22 \frac{\lambda}{\theta} = 1,22 \frac{1,3 \; mm}{7.98 \cdot 10^{-11}} \approx 2.0 \cdot 10^{10} \; mm = 2.0 \cdot 10^{4} \; km
						\end{equation}
				\end{itemize}}

				\begin{alternativas}
					\item $2 \cdot 10^3 \; km$
					\alternativaMarcada $2 \cdot 10^4 \; km$
					\item $2 \cdot 10^5 \; km$
					\item $2 \cdot 10^6 \; km$
					\item $2 \cdot 10^7 \; km$
				\end{alternativas}
			\end{pergunta}
		\end{questao}
	\end{secao}
\end{document}