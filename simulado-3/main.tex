\documentclass{../lista}

\begin{document}
	\cabecalhoAlt
		{Simulado | 2º Intensivo para a OBA}
		{\red{Gabarito}}
		{Gabriel Lucena e Iago Mendes}

	\red{Observação: \begin{itemize}
		\item As alternativas das perguntas deste gabarito não estão na mesma ordem do simulado.
	\end{itemize}}

	\begin{secao}{Questões de Astronomia}
		\begin{questao}{(1 ponto)}
			Some a pontuação de cada item verdadeiro: \\
			
			1. Um observador do hemisfério norte e um observador do hemisfério sul observam a mesma fase da Lua, mas com a imagem invertida
			
			2. O lado oculto da Lua nunca recebe energia do Sol
			
			4. Nós sempre observamos exatamente o mesmo lado da Lua 
			
			8. A Lua Quarto-Minguante nasce ao meio-dia
			
			16. A Lua Quarto-Crescente se põe à meia noite
			
			\red{\begin{itemize}
					\item A fase da lua depende majoritariamente da posição da Lua relativa ao Sol, ou seja, independe do hemisfério observado. No entanto, como dois hemisférios são opostos, a imagem vista de cada um é invertida em relação ao outro. \textbf{Item 1 é verdadeiro.}
					\item O lado oculto da lua recebe energia durante a fase nova, visto que está na direção do Sol. \textbf{Item 2 é falso}
					\item Apesar do efeito de "aprisionamento de maré" que o período orbital é equivalente ao período rotacional, existe uma pequena defasagem cíclica chamada libração que permite a visualização de pequenas áreas do lado oculto da lua. \textbf{O item 4 false}
					\item A Lua quarto-minguante está cerca de 18h (270°) depois do Sol. Ou seja, quando a Lua estiver no horizonte, o Sol vai estár 270° antes da Lua, logo, meia-noite. \textbf{Item 8 falso}
					\item Usando o mesmo raciocínio anterior, a Lua Quarto-Crescente está 6h (90°) depois do Sol. Ou seja, no ocaso da Lua, vai ter passado 6h desde o ocaso do Sol, logo, meia-noite. \textbf{Item 16 verdadeiro.}
					\item Logo, a soma dos itens verdadeiros é 17. 
			\end{itemize}}
			
			\begin{alternativas}
				\item 10
				\alternativaMarcada 17
				\item 18
				\item 20
			\end{alternativas}

		\end{questao}
		\begin{questao}{(1 ponto)}
			Qual dos seguintes anos \textbf{NÃO} é um ano bissexto?
			
			\red{\begin{itemize}
					\item Para um ano ser considerado ano bissexto, deve passar pela verificação na seguinte ordem: \\ 1) É divisível por 400? \\ 2) É divisível por 4 e indivisível por 100? \\ Se, seguindo a ordem, aparecer qualquer resposta afirmativa, o ano é bissexto. Entre as alternativas, a única que é falso para todas as afirmativas é \textbf{2300}.
			\end{itemize}}
		
			\begin{alternativas}
				\item 1956
				\item 2000
				\item 2280
				\alternativaMarcada 2300
			\end{alternativas}
		\end{questao}
		\begin{questao}{(1 ponto)}
			Substitua os nomes dos respectivos astrônomos na ordem correta:
			
			1. O modelo geocêntrico que a Terra se encontra no centro do sistema solar foi proposto por \textbf{nome 1}.
			
			2. \textbf{nome 2} determinou o raio da Terra com medições na cidade de Alexandria e Syene.
			
			3. \textbf{nome 3} propôs o modelo heliocêntrico que o Sol está no centro do sistema solar.
			
			4. \textbf{nome 4} desenvolveu um catálogo de observações astronômicas mais preciso da época, apesar de apoiar o modelo geocêntrico.
			
			\red{\begin{itemize}
					\item Essa é uma questão "sabe ou não sabe" a história, por isso é sempre interessante estar ciente da história da astronomia.
			\end{itemize}}
			
			\begin{alternativas}
				\item Tycho Brahe - Erastóstenes - Copérnico - Ptolomeu
				\item Erastóstenes - Ptolomeu - Copérnico - Tycho Brahe
				\item Copérnico - Ptolomeu - Erastóstenes - Tycho Brahe
				\alternativaMarcada Ptolomeu - Erastóstenes - Copérnico - Tycho Brahe
			\end{alternativas}
		\end{questao}
		\begin{questao}{(1 ponto)}
			\begin{center}
				\includegraphics[height=.75\linewidth]{./img/12.png}
			\end{center}
			
			Some a pontuação para cada um dos itens verdadeiros:
			
			1. A estrela mais brilhante do céu noturno está visível
			
			2. A Grande Nebulosa de Órion está no campo de visão da foto
			
			4. A imagem está completamente no hemisfério sul
			
			8. As estrelas Beteulgeuse e Rigel fazem parte da constelação do Órion
			
			16. A estrela Sírius sempre foi a estrela mais brilhante do céu noturno.
			
			\red{\begin{itemize}
					\item A estrela mais brilhante, Sírius, está visível próximo da extremidade inferior direita. \textbf{Item 1 é verdadeiro}
					\item A Grande Nebulosa de Órion se encontra dentro do asterismo do Órion, que está visível mais próximo do lado superior esquerdo. \textbf{Item 2 é verdadeiro}
					\item O equador passa próximo das estrelas conhecidas como 3 marias, ou seja, parte do hemisfério norte está sendo visto. \textbf{O item 4 é falso.}
					\item Essa pergunta é uma pergunta de conhecimento prévio. \textbf{O item 8 é verdadeiro.}
					\item \textbf{O item 16 é falso.} Já foi encontrado relatos do passado mencionando estrelas mais brilhantes que a estrela Sirius. 
					\item A soma dos itens verdadeiros é 11.
			\end{itemize}}
			
			\begin{alternativas}
				\item 7
				\alternativaMarcada 11
				\item 18
				\item 20
			\end{alternativas}
		\end{questao}
		\begin{questao}{(1 ponto)}
			A seguir temos algumas relações, some a pontuação de cada uma das relações verdadeiros: \\
			
				1. A Terra rotaciona do Oeste para o Leste $\rightarrow$ Sol nasce no Leste e se põe no Oeste.
				
				2. O Sol está no ponto de Áries $\rightarrow$ O Sol está saindo do hemisfério sul e entrando no hemisfério norte.
				
				4. A órbita lunar possui é uma elipse $\rightarrow$ Efeito de libração longitudinal
				
				8. A órbita lunar é inclinada em relação à eclíptica $\rightarrow$ Efeito de libração latitudinal
				
				16. O movimento de precessão $\rightarrow$ A estrela polar de cada hemisfério nunca mudam
				
				\red{\begin{itemize}
						\item O movimento do Sol, na verdade, é apenas a mudança de referencial do sistema solar para um observador na Terra. Como a Terra rotaciona de Oeste para Leste, nós vemos os objetos rotacionando a esfera celeste de Leste para Oeste. \textbf{Item 1 é verdadeiro.}
						\item O Sol está no ponto de áries próximo do dia 21 de março. Ou seja, Outono para nós, do hemisfério sul. E o outono significa transição do verão para o inverno. Logo, o Sol está saindo do hemisfério sul e entrando no hemisfério norte. \textbf{Item 2 é verdadeiro.}
						\item Devido à excentricidade a órbita lunar, existe uma pequena defasagem na observação da Lua, nos permitindo observar uma pequena porcentagem do lado oculto, no eixo longitudinal. \textbf{O item 4 é verdadeiro}
						\item Da mesma forma que o item anterior, a inclinação da órbita lunar ocasiona uma pequena defasagem nos permitindo olhar o lado oculto da lua no eixo latitudinal. \textbf{O item 8 é verdadeiro.}
						\item O movimento de precessão rotaciona o eixo polar ao redor do eixo perpendicular à eclipítica. Logo, o polo está variando e sua estrela polar também. \textbf{O item 16 é falso.}
						\item A soma dos itens verdadeiros é 15
				\end{itemize}}
				
				\begin{alternativas}
					\item 11
					\alternativaMarcada 15
					\item 20
					\item 31
				\end{alternativas}
		\end{questao}
		\begin{questao}{(1 ponto)}
			O Sol tem, aproximadamente, o mesmo tamanho angular da Lua. Sabendo que a distância da Terra ao Sol é 388 vezes maior que a distância da Terra à Lua e o raio da Lua vale $1740 \; km$, qual o tamanho do diâmetro do Sol?
			
			\red{\begin{itemize}
					\item Vamos utilizar a equação do tamanho angular de qualquer objeto: 
					\begin{equation} 
						\tan \alpha = \frac{R_{objeto}}{d}
					\end{equation}
					\item Também vamos utilizar a consideração que o tamanho angular do Sol e da Lua são iguais:
					\begin{equation}
						\alpha_{Sol} = \alpha_{Lua} \rightarrow \tan \alpha_{Sol} = \tan \alpha_{Lua}
					\end{equation}
					\item Então vamos desenvolver as equações:
					\begin{equation}
						\frac{R_{sol}}{ d_{Terra-Sol}} = \frac{R_{Lua}}{d_{Terra-Lua}} \\
						R_{Sol} = R_{Lua} \cdot \frac{d_{Terra-Sol}}{d_{Terra-Lua}}
						R_{sol} = R_{Lua} \cdot 388 = 675120 \; km
					\end{equation}
					\item Como o diâmetro é o dobro do raio, então o diâmetro do Sol equivale:
					\begin{equation}
						D_{Sol} = 1350240 \; km
					\end{equation}
			\end{itemize}}
			
			\begin{alternativas}
				\item $675120 \; km$
				\item $337560 \; km$
				\alternativaMarcada $1350240 \; km$
				\item $4050720 \; km$
			\end{alternativas}
			
		\end{questao}
		\begin{questao}{(1 ponto)}
			Um satélite geoestacionário é um satélite que possui período orbital igual ao período de rotação da Terra. Calcule a \textbf{altura aproximada} da órbita geoestacionária em relação à superfície da Terra. Utilize a terceira lei de Kepler e os seguintes dados:
			
			\begin{equation}
				M_{Terra} = 6 \cdot 10^{24} \; kg \\
				G = 6.67 \cdot 10^{-11} \; N m^{2} / kg^{2} \\
				R_{Terra} = 6371 \; km
			\end{equation}
		
			\red{\begin{itemize}
					\item Vamos utilizar a 3ª Lei de Kepler:
					\begin{equation}
						\frac{ a^{3} }{ T^{2} } = \frac{ G M_{Terra} }{ 4 \pi^{2} }
					\end{equation}
					\item Sabendo que $T \approx 24 \; h$ para a órbita estacionária e $a$ é o semi-eixo maior dessa órbita, vamos susbtituir os valores:
					\begin{equation}
						\frac{ a^{3} }{ (24 \; h)^{2} } = \frac{ 6.7 \cdot 10^{-11} \cdot 6 \cdot 10^{24} }{ 4 \pi^{2} } \; m^{3}/s^{2} \\
						a \approx 42300 km
					\end{equation}
				\item Logo, a \textbf{altura aproximada} em relação à superfície é, nesse caso, o semi-eixo maior subtraindo o raio da Terra.
				\begin{equation}
					h = a - R_{Terra} \\
					h \approx 35900 \; km
				\end{equation}
			\end{itemize}}
			
			\begin{alternativas}
				\item $48680 \; km$
				\item $42300 \; km$
				\alternativaMarcada $35900 \; km$
				\item $29500 \; km$
			\end{alternativas}
		\end{questao}
	\end{secao}

	\begin{secao}{Questões de Astronáutica}
		\begin{questao}{(1 ponto) [OBA 2008 Adaptada]}
			Uma empresa privada dos EUA está desenvolvendo um avião espacial (SpaceShipTwo) no qual turistas viajarão ao espaço em um vôo suborbital de 15 a 20 minutos. Durante a fase do vôo fora da atmosfera da Terra os turistas conseguirão ver a Terra da mesma forma que os astronautas a vêem em seus vôos orbitais e da Estação Espacial Internacional. Conforme mostrado na imagem ao lado, obtida do espaço, é possível ver claramente a curvatura da Terra (Aparentemente a Terra não é plana!). Analisando a imagem e usando a geometria e trigonometria que você aprendeu na escola é possível estimar a altitude da qual ela foi tirada. Neste caso, o comprimento estimado para o campo de visão horizontal é de 1.200 km.
			
			\begin{center}
				\includegraphics[height=.4\linewidth]{./img/13.png} \includegraphics[height=.4\linewidth]{./img/14.png}
			\end{center}
			
			a) Com o uso da trigonometria podemos determinar outras informações a partir da imagem. Sabendo-se que o ângulo de visão da câmara fotográfica é de 45 graus na horizontal, determine a distância \textbf{d} do astronauta que tirou a foto até o horizonte da Terra. \\
			
			b) A distância d de um ponto qualquer acima da superfície da Terra até o horizonte é dada por $d = \sqrt{2 R h + h^{2}}$, onde R é o raio da Terra e h a altura onde foi feita a imagem. Determine a altura h da órbita de onde foi feita a imagem acima. Use a distância d obtida no item anterior. Utilize a calculadora para qualquer cálculo.
			
			\red{\begin{itemize}
					\item A distância d pode ser calculada utilizando uma composição do triângulo da imagem. Se você desenhar a altura do triângulo em relação ao lado de $1200 \; km$, você terá dois triângulos semelhantes, como se tivesse dividido o triângulo maior em duas partes iguais. Esse triângulo menor é um triângulo retângulo, que você poderá utilizar a relação trigonométrica tangente da seguinte maneira:
					\begin{equation}
						\tan (\frac{45^\circ}{2}) = \frac{\frac{1200 \; km}{2}}{d} \\
						d = \frac{600 \; km}{\tan(22.5^\circ) } = 1500 \; km
					\end{equation}
					\item Utilizando a equação do item b, você obtem uma equação do segundo grau com incógnita $h$
					\begin{equation}
						d = \sqrt{2 R h + h^{2}} \\
						d^{2} = 2 R h +  h^2 \\
						0 = 1 \cdot h^2 + 2R \cdot h - d^2
					\end{equation}
					\item Nesse momento, você pode utilizar a \textbf{Equação de Baskhara}. Você encontrará dois valores como resultado de $h$, porém um dos resultados é negativo. Como estamos falando de \textbf{distâncias}, esse resultado é fisicamente impossível, então o outro resultado é nossa resposta.
					\begin{equation}
						h = 174 \; km
					\end{equation}
				\end{itemize}}
			
			\begin{alternativas}
				\item a. $1500 \; km$ e b. $300 \; km$
				\item a. $1200 \; km$ e b. $174 \; km$
				\alternativaMarcada a. $1500 \; km$ e b. $174 \; km$
				\item a. $1200 \; km$ e b. $300 \; km$
			\end{alternativas}
		\end{questao}
		\begin{questao}{(1 ponto) [OBA 2016 Adaptada]}
			De uma maneira simplificada um satélite de sensoriamento remoto pode ser entendido como uma “máquina fotográfica” que, do espaço, obtém imagens da Terra. A partir dessas imagens é possível monitorar e medir vários fenômenos que ocorrem na superfície terrestre, incluindo queimadas e desmatamento. É importante ressaltar, contudo, que a identificação das queimadas é feita a partir da captação da energia emitida pelo material orgânico em chamas, que ocorre, principalmente na faixa de comprimento de ondas entre 3,7µm e 4,1µm (1 µm = 10-6 m) do espectro eletromagnético, conhecida como termal-média. Sabe-se que quanto maior a temperatura da chama, maior é a emissão de energia. O desmatamento, por sua vez, é identificado a partir da radiação solar refletida em uma faixa de comprimento de onda entre 0,4 µm e 3,0 µm. Ao se analisar a radiação solar refletida pelos tipos de superfície nos diversos comprimentos de onda da radiação solar observa-se que a água (rios, lagos e mares) reflete menos energia solar quando comparada ao solo sem cobertura vegetal e ao solo com cobertura vegetal. Além disso, o solo exposto e a vegetação refletem diferentemente em todos os comprimentos de onda, o que permite sua diferenciação. Por se tratarem de fenômenos físicos distintos (emissão e reflexão) o satélite precisa possuir mais de uma câmera imageadora para monitorar o desmatamento e as queimadas. De modo similar a uma máquina fotográfica digital, as imagens obtidas pelos sensores de um satélite são transformadas em píxeis. Cada imagem é composta de milhões de píxeis. O pixel é o menor elemento da imagem, ao qual é possível atribuir uma tonalidade, cujo valor numérico varia entre zero e 255. Um pixel com valor zero significa que ele recebeu quase nenhuma radiação proveniente da superfície terrestre, sendo então representado pela cor preta. No outro extremo o valor 255 corresponde à cor branca e indica que o sensor recebeu a máxima quantidade de radiação da superfície terrestre. Entre zero e 255 há 254 tons de cinza do mais claro ao mais escuro. O normal é uma imagem com píxeis de diversas tonalidades de cinza, da mais clara (tendendo ao branco) à mais escura (tendendo ao negro). \\
			
			A partir dessas informações, some a pontuação em cada uma das seguintes sentenças: \\
			
			1. A partir de variações de tonalidade de cinza obtidas nas imagens dos satélites, os cientistas identificam regiões de queimadas e de desmatamento
			
			2. A presença de nuvens não atrapalha a detecção de queimadas e de desmatamento.
			
			4. Uma área queimada, depois do fogo extinto, irá refletir mais radiação solar do que antes, quando havia cobertura vegetal, e por isso, será representada por “píxeis” claros
			
			8. Quanto maior a temperatura da área sendo queimada, mais claros serão os píxeis que representam a imagem dessa área.
			
			16. Muitos píxeis de uma imagem de uma câmera satelital, destinada ao monitoramento de queimadas, apresentam valores numéricos próximos de 255. Isso significa a detecção de uma queimada.
			
			\red{\begin{itemize}
					\item O \textbf{item 1 é verdadeiro}, pois, como explicado no texto, padrões de cores são associados às condições de desmatamento, queimada, vegetação intacta e rios.
					\item O \textbf{item 2 é falso}, pois a presença de nuvens impede que a radiação por emissão e reflexão cheguem à câmera.
					\item O \textbf{item 4 é falso}, pois vai contra a constatação do texto.
					\item O textbf{item 8 é verdadeiro}, pois, como explicado no texto, quanto mais quente as chamas, maior q emissão no espectro termalmédia.
					\item O \textbf{item 16 é verdadeiro}, como dito no item anterior, quanto maior a intensidade das chamas, maior a temperatura e mais claros os píxeis associados a imagem dessa área.
			\end{itemize}}
			
			\begin{alternativas}
				\item 4
				\item 9
				\item 17
				\alternativaMarcada 25
			\end{alternativas}
		\end{questao}
		\begin{questao}{(1 ponto) [OBA 2018 Adaptada]}
			Com o desenvolvimento da astronáutica está cada vez mais fácil colocarmos telescópios em órbita. Contudo, alguns, como o SOHO (Solar and Heliospheric Observatory = Observatório Solar e Heliosférico), precisam girar ao redor do Sol no mesmo período que a Terra e ficar entre o Sol e a Terra, pois precisa observar o Sol 24h/dia. Mas pela terceira lei de Kepler , ou seja, quanto menor a distância ao Sol, menor será o período e viceversa. Logo, não seria possível colocar o SOHO e outros satélites para girarem ao redor do Sol, com o mesmo período da Terra estando num lugar diferente da Terra. Mas o italiano JosephLouis de Lagrange, em 1772, descobriu que há cinco pontos, chamados pontos Lagrangianos, num sistema Terra-Sol, ou Terra-Lua, ou Solplaneta, que são “especiais”. O ponto L1 fica na linha Terra-Sol, entre Terra e Sol e um observatório ali colocado move-se com o mesmo período da Terra, tal com faz o SOHO, o qual nunca é eclipsado pela Lua e recebe sempre a mesma irradiação do Sol. Veja a figura ao lado. O ponto L2 fica depois do cone de sombra (umbra) da Terra, será o local de posicionamento do Telescópio Espacial James Webb e terá período de translação igual ao da Terra. Os pontos L4 e L5 ficam sobre a órbita da Terra e são localizados por um triângulo equilátero com aresta igual à distância Terra-Sol.
			
			\begin{center}
				\includegraphics[height=.75\linewidth]{./img/15.png}
			\end{center}
		
			a) Considere que M1 seja a massa do Sol, M2 a massa da Terra, R a distância Terra-Sol e r a distância da Terra aos pontos Lagrangianos L1 e L2 (são simétricos em relação a M2). Pode-se demonstrar que r é dado por: $ r = \sqrt[3]{\frac{M_{1}}{3 M_{2}}} R = 1.4784 \cdot 10^{8} \; km$. Sabendo que a distância média à Lua é de $384000 \; km$, calcule quantas vezes $r$ está mais distante que a órbita da Lua.
			
			b) Conforme explicado, a vantagem dos pontos L1, L2 e L3 é que mesmo
			estando à diferentes distâncias da Terra ao Sol, ainda assim, satélites ali colocados teriam o mesmo período de translação da Terra ao redor do Sol, isto é, 365,25 dias. Qual seria o período de translação de satélites colocados nos pontos Lagrangianos L4 e L5?
			
			\red{\begin{itemize}
					\item O item a, basta dividir a seguinte razão:
					\begin{equation}
						\frac{r}{ 384000 \; km } = \frac{ 1,4784 \cdot 10^{8} }{ 384000 } = 385
					\end{equation}
					\item Uma das condições necessárias para o estado estacionário dos pontos de lagrange na órbita relativa aos dois corpos mais massivos, é a necessidade de período orbital equivalente. Logo, o período de tralação de satélites nos pontos de lagrange L4 e L5 é \textbf{o mesmo período da Terra} 
			\end{itemize}}
			
			\begin{alternativas}
				\item a) 200 b) Metade do período da Terra
				\item a) 385 b) Metade do período da Terra
				\item a) 200 b) O mesmo período da Terra
				\alternativaMarcada a) 385 b) O mesmo período da Terra
			\end{alternativas}
	
		\end{questao}
	\end{secao}

	\begin{secao}{Questões Avançadas}
		\begin{questao}{(1 ponto)}
			Um fenômeno muito conhecido é o da ``laçada de Marte", em que o planeta Marte subitamente muda sua direção de deslocamento no céu, e quando acompanhado por vários dias parece se locomover formando um laço no céu.

			\begin{pergunta}{(1 ponto)}
				Quais planetas, além de Marte, reproduzem o mesmo fenômeno de modo que possamos observá-los em uma noite de céu limpo?

				\red{\begin{itemize}
						\item Todos os planetas reproduzem esse fenômeno. Então, a pegadinha da questão é você marcar somente os planetas que são observáveis à noite, excluindo assim os planetas inferiores (Mercúrio e Vênus), os quais estão sempre próximos ao Sol na Esfera Celeste.
				\end{itemize}}

				\begin{multicols}{3}
					\begin{alternativas}
						\item[$(\quad)$] Mercúrio
						\item[$(\quad)$] Vênus
						\alternativaMarcada Júpiter
						\alternativaMarcada Saturno
						\alternativaMarcada Urano
						\alternativaMarcada Netuno
					\end{alternativas}
				\end{multicols}
			\end{pergunta}
		\end{questao}

		\begin{questao}{(1 ponto) [USAAAO 2021 adaptada]}
			O cometa C/2020 F3 (NEOWISE) atingiu o periélio pela última vez em 3 de julho de 2020. O cometa NEOWISE tem um período orbital de $\approx 4.400$ anos e sua excentricidade é de $0,99921$.

			\begin{pergunta}{(1 ponto)}
				Qual é a distância do periélio do cometa NEOWISE, em $UA$?

				\red{\begin{itemize}
					\item Usando a 3ª Lei de Kepler com unidades do Sistema Solar (anos, $UA$ e massas solares), nós temos:
						\begin{equation}
							\frac{T^2}{a^3} = 1 \quad \therefore \quad a = \sqrt[3]{T^2} = \sqrt[3]{4400^2} \approx 268,5 \; UA
						\end{equation}
					\item Agora, basta calcular a distância do periélio usando a geometria de elipses:
						\begin{equation}
							P = a(1-e) = 268,5 (1-0,99921) \approx 0.212 \; UA
						\end{equation}
				\end{itemize}}

				\begin{alternativas}
					\item $0,0123 \; UA$
					\alternativaMarcada $0,212 \; UA$
					\item $2,69 \; UA$
					\item $26,8 \; UA$
				\end{alternativas}
			\end{pergunta}
		\end{questao}

		\begin{questao}{(1 ponto)}
			Deneb é uma estrela de tipo espectral A2 cuja magnitude aparente na banda V é de 1,25. Certa noite Deneb se divide em 2 novas estrelas com a mesma temperatura da inicial. \\ \\
			Dado: \\
			$\log(2) \approx 0,3$

			\begin{pergunta}{(1 ponto)}
				Qual a nova magnitude aparente na banda V do sistema?

				\red{\begin{itemize}
					\item Como o volume de cada uma das novas estrelas deve ser metade de Deneb, temos a seguinte relação entre os raios das novas estrelas ($R'$) e de Deneb ($R_0$):
						\begin{equation}
									V'=\frac{V_0}{2} \quad \therefore \quad \frac{4\pi R'^3}{3}=\frac{1}{2} \frac{4 \pi R_0^3}{3} \quad \therefore \quad \frac{R'}{R_0}=\sqrt[3]{\frac{1}{2}}
						\end{equation}
					\item Lembrando que $T'=T_0$, podemos usar a equação de Stefan-Boltzmann para encontrar a razão entre as luminosidades:
						\begin{equation}
							\frac{L'}{L_0}=\frac{4 \pi R'^2 \sigma T'^4}{4 \pi R_0^2 \sigma T_0^4}=\left(\frac{R'}{R_0}\right)^2=2^{-\frac{2}{3}}
						\end{equation}
					\item Calculando a razão dos fluxos recebidos, temos:
						\begin{equation}
							\frac{F'}{F_0}=\frac{2L'}{L_0}=2 \cdot 2^{-\frac{2}{3}}=\sqrt[3]{2}
						\end{equation}
					\item Finalmente, usando a relação de Pógson, temos:
						\begin{equation}
							m'-m_0=2,5 \log \left(\frac{F_0}{F'}\right) \quad \therefore \quad m'=2,5 \log \left(\frac{F_0}{F'}\right)+m_0\\
							\therefore \quad m'=2,5 \log \left(2^{-\frac{1}{3}}\right)+1,25=-\frac{2,5 \cdot \log (2)}{3}+1,25 \\
							\therefore \quad m'=-0,25 +1,25 =1,0
						\end{equation}
				\end{itemize}}

				\begin{alternativas}
					\item $1,25$
					\item $2,5$
					\alternativaMarcada $1,0$
					\item $2,0$
				\end{alternativas}
			\end{pergunta}
		\end{questao}

		\begin{questao}{(1 ponto) [USAAAO 2020 adaptada]}
			Em abril de 2020, o \textit{Event Horizon Telescope} divulgou a primeira imagem do buraco negro supermassivo da galáxia \textit{M87}. O buraco negro tem um diâmetro de aproximadamente $270 \; UA$ e está localizado a uma distância de $16,4 \; Mpc$.

			\begin{pergunta}{(1 ponto)}
				No comprimento de onda observado de $1,3 \; mm$, qual é a linha de base mínima aproximada, ou diâmetro efetivo, necessária para a imagem do buraco negro?

				\red{\begin{itemize}
					\item Calculando a resolução angular necessária para observar o buraco negro de \textit{M87}, temos:
						\begin{equation}
							\theta = \frac{270 \; UA}{16,4 \; Mpc} = \frac{270 \; UA}{16,4 \cdot 10^6 \cdot 206.265 \; UA} \approx 7.98 \cdot 10^{-11} \; rad
						\end{equation}
					\item Agora, basta usar a equação de resolução angular de telescópios para encontrar o diâmetro efetivo:
						\begin{equation}
							\theta = 1,22 \frac{\lambda}{D} \quad \therefore \quad D = 1,22 \frac{\lambda}{\theta} = 1,22 \frac{1,3 \; mm}{7.98 \cdot 10^{-11}} \approx 2.0 \cdot 10^{10} \; mm = 2.0 \cdot 10^{4} \; km
						\end{equation}
				\end{itemize}}

				\begin{alternativas}
					\item $2 \cdot 10^3 \; km$
					\alternativaMarcada $2 \cdot 10^4 \; km$
					\item $2 \cdot 10^5 \; km$
					\item $2 \cdot 10^6 \; km$
					\item $2 \cdot 10^7 \; km$
				\end{alternativas}
			\end{pergunta}
		\end{questao}

		\begin{questao}{(1 ponto) [Seletiva OBA Presencial 2016-17 adaptada]}
			A paralaxe heliocêntrica de Canopus, segundo os dados do satélite Hipparcos, vale $10,42$ milisegundos de arco ($mas$). \\ \\
			Dado: magnitude aparente de Canopus = $-0,72$

			\begin{pergunta}{(0,5 ponto)}
				Utilize essa informação e o módulo da distância para calcular a magnitude absoluta de Canopus.

				\red{\begin{itemize}
					\item Usando a paralaxe, podemos calcular a distância de Canopus:
						\begin{equation}
							\pi = \frac{1}{r} \quad \therefore \quad r = \frac{1}{\pi}=\frac{1}{10,42 \cdot 10^{-3}} \approx 95,97 \, pc
						\end{equation}
					\item Agora, com o módulo de distância, temos:
						\begin{equation}
							m-M=5 \log r -5 \quad \therefore \quad M = 5 \log r -5-m=5 \cdot \log (95,97)-5-0.72 \approx -5,63
						\end{equation}
				\end{itemize}}

				\begin{alternativas}
					\alternativaMarcada $M = -5,63$
					\item $M = -0,72$
					\item $M = -1,44$
					\item $M = -2,82$
				\end{alternativas}
			\end{pergunta}

			\begin{pergunta}{(0,5 ponto)}
				Aproximadamente, quantas vezes ela é mais luminosa do que o Sol? \\
				Dado: magnitude absoluta do Sol = $+4,80$

				\red{\begin{itemize}
					\item Usando a relação de Pógson com as magnitudes absolutas, temos:
						\begin{equation}
							M_S-M_C = 2,5 \log \left(\frac{L_C}{L_S}\right) \quad \therefore \quad \frac{L_C}{L_S} = 10^{\frac{M_S-M_C}{2,5}} \\
							\therefore \quad \frac{L_C}{L_S} \approx 1,5 \cdot 10^4 = 15.000
						\end{equation}
				\end{itemize}}

				\begin{alternativas}
					\alternativaMarcada $\approx 15$ mil vezes
					\item $\approx 30$ mil vezes
					\item $\approx 5$ mil vezes
					\item $\approx 50$ mil vezes
				\end{alternativas}
			\end{pergunta}
		\end{questao}
	\end{secao}
\end{document}