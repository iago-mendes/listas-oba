\documentclass[a4paper, 12pt]{article}
\usepackage[top=1cm, bottom=2.1cm, left=2cm, right=2cm]{geometry}
\usepackage[utf8]{inputenc}
\usepackage{graphicx, caption}
\usepackage{float}
\usepackage{amsmath, amsfonts, amssymb, esint}
\usepackage{hyperref}
\usepackage{multicol}
\usepackage{wallpaper}

\CenterWallPaper{1}{./img/background.png}

\hypersetup{
    colorlinks=true,
    linkcolor=blue,
    filecolor=magenta,      
    urlcolor=cyan,
}

\newcommand{\cabecalho}[4]
{
	\begin{figure}
		\centering
		\href{https://ligaolimpicadeastronomia.com.br/}{\includegraphics[scale=0.6]{./img/logos.png}}
	\end{figure}
	
	\begin{center}
		\begin{large}
			\textbf{#1}	
		\end{large}
			\linebreak Listas OBA (Nível 4) -- #2ª Lista
			\linebreak #3
		\end{center}
	
	\begin{flushright}
		Material elaborado por \textbf{#4}
	\end{flushright}
}

\begin{document}
	\cabecalho{Questões}{5}{Sistema Solar e História}{Iago Mendes}
	
	\begin{flushleft}
	\begin{itemize}
		\item \textbf{Questão 1) (1 ponto)}
		
		\item \textbf{Questão 2) (1 ponto)}
		
		\item \textbf{Questão 3) (1 ponto)} 
			
		\item \textbf{Questão 4) (1 ponto)} 
				
		\item \textbf{Questão 5) (1 ponto)} 
			
	\end{itemize}
	\end{flushleft}
	\begin{flushright}
		\begin{large}
			Bons estudos!
		\end{large}
	\end{flushright}
\end{document}