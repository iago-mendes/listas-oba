\documentclass[a4paper, 12pt]{article}
\usepackage[top=1cm, bottom=2.1cm, left=2cm, right=2cm]{geometry}
\usepackage[utf8]{inputenc}
\usepackage{graphicx, caption}
\usepackage{float}
\usepackage{amsmath, amsfonts, amssymb, esint}
\usepackage{hyperref}
\usepackage{multicol}
\usepackage{wallpaper}
\usepackage{array}

\CenterWallPaper{1}{./img/background.png}

\hypersetup{
    colorlinks=true,
    linkcolor=blue,
    filecolor=magenta,      
    urlcolor=cyan,
}

\newcommand{\cabecalho}[4]
{
	\begin{figure}
		\centering
		\href{https://ligaolimpicadeastronomia.com.br/}{\includegraphics[scale=0.6]{./img/logos.png}}
	\end{figure}
	
	\begin{center}
		\begin{large}
			\textbf{#1}	
		\end{large}
			\linebreak Listas OBA (Nível 4) -- #2ª Lista
			\linebreak #3
		\end{center}
	
	\begin{flushright}
		Material elaborado por \textbf{#4}
	\end{flushright}
}

\newcolumntype{M}[1]{>{\centering\arraybackslash}m{#1}}

\begin{document}
	\cabecalho{Questões}{5}{Sistema Solar e História}{Iago Mendes}
	
	\begin{itemize}
		\item \textbf{Questão 1) (1 ponto) (0,2 cada acerto)} Coloque o nome do astro respectivo às seguintes descrições:
			\begin{center} \begin{tabular}{|M{0.2\textwidth}|M{0.7\textwidth}|}
				\hline
				.......... & Possui uma amplitude térmica altíssima, variando de $427^{\circ}C$ até $-137^{\circ}C$. Sua superfície se assemelha com a da Lua. Leva cerca de 59 dias para completar uma volta em torno de si e 88 dias para realizar uma revolução completa em torno do Sol. \\ \hline
				.......... & Astro com maior temperatura do Sistema Solar devido ao efeito estufa. Pode ser visto no céu noturno em certas épocas do ano (sempre próximo ao Sol). Seu dia é mais longo do que seu ano. \\ \hline
				.......... & Um dos satélites naturais mais famosos no Sistema Solar, sendo o segundo maior. É o único astro -- além da Terra -- onde já foram encontradas evidências concretas da existência de corpos líquidos estáveis na superfície. Foi descoberto em 1655 pelo astrônomo Christiaan Huygens. \\ \hline
				.......... & Astro com densidade inferior a da água. Foi estudado pela missão Cassini–Huygens. É o último planeta facilmente vísivel a olho nu da Terra e o que possui o maior número de satélites naturais encontrados.\\ \hline
				.......... & Possui coloração azulada devido à existência de gás metano em sua atmosfera. Seus movimentos são curiosos, pois, além de ser um dos 2 planetas que rotacionam de Leste para Oeste, possui seu eixo de rotação ``deitado" em relação ao plano de translação. Possui um anel constituído por rochas.\\ \hline
			\end{tabular} \end{center}
		
		\item \textbf{Questão 2) (1 ponto)}
		
		\item \textbf{Questão 3) (1 ponto)} 
			
		\item \textbf{Questão 4) (1 ponto)} 
				
		\item \textbf{Questão 5) (1 ponto)} 
			
	\end{itemize}

	\begin{flushright}
		\begin{large}
			Bons estudos!
		\end{large}
	\end{flushright}
\end{document}