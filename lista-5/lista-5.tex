\documentclass[a4paper, 12pt]{article}
\usepackage[top=1cm, bottom=2.1cm, left=2cm, right=2cm]{geometry}
\usepackage[utf8]{inputenc}
\usepackage{graphicx, caption}
\usepackage{float}
\usepackage{amsmath, amsfonts, amssymb, esint}
\usepackage{hyperref}
\usepackage{multicol}
\usepackage{wallpaper}
\usepackage{array}

\CenterWallPaper{1}{./img/background.png}

\hypersetup{
    colorlinks=true,
    linkcolor=blue,
    filecolor=magenta,      
    urlcolor=cyan,
}

\newcommand{\cabecalho}[4]
{
	\begin{figure}
		\centering
		\href{https://ligaolimpicadeastronomia.com.br/}{\includegraphics[scale=0.6]{./img/logos.png}}
	\end{figure}
	
	\begin{center}
		\begin{large}
			\textbf{#1}	
		\end{large}
			\linebreak Listas OBA (Nível 4) -- #2ª Lista
			\linebreak #3
		\end{center}
	
	\begin{flushright}
		Material elaborado por \textbf{#4}
	\end{flushright}
}

\newcolumntype{M}[1]{>{\centering\arraybackslash}m{#1}}

\begin{document}
	\cabecalho{Questões}{5}{Sistema Solar e História}{Iago Mendes}
	
	\begin{itemize}
		\item \textbf{Questão 1) (1 ponto) (0,2 cada acerto)} Coloque o nome do astro respectivo às seguintes descrições:
			\begin{center} \begin{tabular}{|M{0.2\textwidth}|M{0.65\textwidth}|}
				\hline
				.......... & Possui uma amplitude térmica altíssima, variando de $427^{\circ}C$ até $-137^{\circ}C$. Sua superfície se assemelha com a da Lua. Leva cerca de 59 dias para completar uma volta em torno de si e 88 dias para realizar uma revolução completa em torno do Sol. \\ \hline
				.......... & Astro com maior temperatura do Sistema Solar devido ao efeito estufa. Pode ser visto no céu noturno em certas épocas do ano (sempre próximo ao Sol). Seu dia é mais longo do que seu ano. \\ \hline
				.......... & Um dos satélites naturais mais famosos no Sistema Solar, sendo o segundo maior. É o único astro -- além da Terra -- onde já foram encontradas evidências concretas da existência de corpos líquidos estáveis na superfície. Foi descoberto em 1655 pelo astrônomo Christiaan Huygens. \\ \hline
				.......... & Astro com densidade inferior a da água. Foi estudado pela missão Cassini–Huygens. É o último planeta facilmente vísivel a olho nu da Terra e o que possui o maior número de satélites naturais encontrados. Seu anel é feito de gelo. \\ \hline
				.......... & Possui coloração azulada devido à existência de gás metano em sua atmosfera. Seus movimentos são curiosos, pois, além de ser um dos 2 planetas que rotacionam de Leste para Oeste, possui seu eixo de rotação ``deitado" em relação ao plano de translação. Possui um anel constituído por rochas.\\ \hline
			\end{tabular} \end{center}
		
		\item \textbf{Questão 2) (1 ponto) (0,2 cada acerto)} Coloque o nome do astro respectivo às seguintes descrições:
			\begin{center} \begin{tabular}{|M{0.2\textwidth}|M{0.65\textwidth}|}
				\hline
				.......... & Possui faixas brancas e marrons em sua atmosfera, além de uma tempestade que gera o que chamamos de \textit{A Grande Mancha Vermelha}. Seu anel é rochoso e seus satélites naturais mais famosos foram estudados por \textit{Galileo Galilei}. \\ \hline
				.......... & Lugar do \textit{Monte Olímpo} (maior vulcão do Sistema Solar). Possui 2 satélites naturais: Phobos e Deimos. Está a uma distância de aproximadamente $1,5 \, UA$ do Sol. \\ \hline
				.......... & Possui afélio no \textit{Cinturão de Kuiper} e periélio a cerca de $0,6 \, UA$ do Sol. Composto majoritariamente por rochas, poeira, e gases congelados. Seu período orbital é aproximadamente $76$ anos. \\ \hline
				.......... & Descoberto por \textit{Galileo Galilei} em 1610. É o maior e mais massivo satélite natural do Sistema Solar. Pode ser observado próximo ao planeta em torno de qual orbita, estando algumas vezes alinhado com outros 3 astros.  \\ \hline
				.......... & Suas primeiras imagens de qualidade foram tiradas pela sonda \textit{New Horizons} em 2015. É um objeto transnetuniano, pertencendo ao \textit{Cinturão de Kuiper}. Em 2006, devido a uma redefinição feita pela \textit{IAU}, sua classificação foi alterada. \\ \hline
			\end{tabular} \end{center}
		
		\item \textbf{Questão 3) (1 ponto)} Marque a segunda coluna em relação à primeira, indicando a classificação correta de cada astro.
			\begin{multicols}{2}
				\vfill\null \vfill\null
				\begin{flushleft}
					(1) Planeta anão \linebreak
					(2) Satélite natural \linebreak
					(3) Cometa de curto período \linebreak
					(4) Cometa de longo período
				\end{flushleft}
				\vfill\null \vfill\null
				\columnbreak
				\begin{itemize}
					\item[$(\quad)$] Astro popularmente conhecido como Neowise, mas com nome oficial C/2020 F3. Foi descoberto em 2020. Está localizado na Nuvem de Oort. Possui um período orbital de aproximadamente $6.766$ anos.
					\item[$(\quad)$] Astro chamado Ceres. Foi descoberto em 1801. Está localizado no Cinturão de Asteroides. Possui um período orbital de aproximadamente $5$ anos.
					\item[$(\quad)$] Astro popurlamente conhecido como Halley, mas com nome oficial 1P/Halley. Foi descoberto em 1758. Está localizado no Cinturão de Kuiper. Possui um período orbital de aproximadamente $76$ anos.
					\item[$(\quad)$] Astro chamado Encélados. Foi descoberto em 1789. Está localizado na faixa de planetas gasosos. Possui um período orbital de aproximadamente 33 horas.
				\end{itemize}

			\end{multicols}
		\newpage
		\item \textbf{Questão 4) (1 ponto)} Marque a segunda coluna em relação à primeira, indicando qual astrônomo foi responsável pelos feitos descritos.
			\begin{multicols}{2}
				\vfill\null \vfill\null
				\begin{flushleft}
					(1) Galileo Galilei \linebreak
					(2) Sir Isaac Newton \linebreak
					(3) Johannes Kepler
				\end{flushleft}
				\vfill\null \vfill\null
				\columnbreak
				\begin{itemize}
					\item[$(\quad)$] Nasceu em 1571. Criou 3 leis que tinham como objetivo descrever o movimento dos planetas. Desenvolveu um tipo de telescópio refrator, em que as lentes objetiva e ocular são convergentes.
					\item[$(\quad)$] Nasceu em 1564. Desenvolveu um tipo de telescópio refrator, em que as lentes objetiva e ocular são convergente e divergente, respectivamente. Descobriu os 4 satélites naturais principais do maior planeta do Sistema Solar.
					\item[$(\quad)$] Nasceu em 1643. Criou 3 leis que são usadas como base da Dinâmica, divisão da Mecânica. Desenvolveu um tipo de telescópio refletor, em que os espelhos principal e secundário são côncavo e plano, respectivamente.
				\end{itemize}
			\end{multicols}

		\item \textbf{Questão 5) (1 ponto)} 
			
	\end{itemize}

	\begin{flushright}
		\begin{large}
			Bons estudos!
		\end{large}
	\end{flushright}
\end{document}